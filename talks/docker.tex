\documentclass{beamer}

\mode<presentation> {
  \usetheme{Madrid}
}

\title[Docker]{Docker Intro} % The short title appears at the bottom of every slide, the full title is only on the title page

\author{Radu Ciorba} % Your name
\institute[Spyhce] % Your institution as it will appear on the bottom of every slide, may be shorthand to save space
{
  Spyhce \\ % Your institution for the title page
  \medskip
  \textit{radu.ciorba@spyhce.com} % Your email address
}
\date{\today} % Date, can be changed to a custom date

\begin{document}

\begin{frame}
  \titlepage % Print the title page as the first slide
\end{frame}

\begin{frame}
  \frametitle{Context}
  \begin{itemize}
  \item started playing arround with docker in 2013
  \item 2014 for dev stuff + to deploy a ruby app at HackTM
  \item 2015 using it in automated tests + wrote a trivial init system
  \end{itemize}
\end{frame}

\begin{frame}
  \frametitle{What is Docker?}
  \begin{itemize}
  \item Container runtime.
  \item A toolchain to package applications and distribute them.
  \end{itemize}
\end{frame}

\begin{frame}
  \frametitle{What's a container}
  \begin{itemize}
  \item Looks like a lightweight VM (if you squint hard enought)
  \item Containers: Kernel level isolation (PID, users, networking, ...)
  \item FreeBSD Jails 2000, Solaris Zones 2004, LXC 2008
  \end{itemize}
\end{frame}

\begin{frame}
  \frametitle{Why a new tool when we have LXC?}
  \begin{center}
  LXC wants to solve the problem of lightweight VMs. \par
  \noindent
  Docker wants to solve the problem of application packaging and deployment.
  \end{center}
\end{frame}

\begin{frame}
  \frametitle{The docker way}
  \begin{itemize}
  \item Don't treat it like a VM but an Application Container.
  \item Only run one ``App'' per container.
  \item Have that app log to stdout, so Docker can collect the logs.
  \end{itemize}
\end{frame}

\begin{frame}
  \frametitle{Live Demo}
\end{frame}

\begin{frame}
  \frametitle{Persisting Data}
  \begin{itemize}
  \item Containers are ephemeral.
  \item You can have 'VOLUMES' for a container.
  \item Use a dedicated DATA container and the --volumes-from flag
  \item Or just bind mount a volume - very useful in development
  \end{itemize}
\end{frame}

\begin{frame}
  \frametitle{Networking}
  \begin{itemize}
  \item By default docker creates a docker0 bridge and you can map ports
  \item You can use another container's network
  \item You can use the host's network
  \end{itemize}
\end{frame}

\begin{frame}
  \frametitle{What's nice about Docker}
  \begin{itemize}
  \item Building versioned images and sharing them!
  \item Imagine you build a version, deploy it to QA, test it there, take that exact environment (code, os, libraries) and put it in PROD.
  \item Imagine you could ensure your dev environment is EXACTLY the same as prod?
  \item Once an image is built, you don't care what's in there, you just run it.
  \end{itemize}
\end{frame}

\begin{frame}
  \frametitle{When developing}
  \begin{itemize}
  \item Bind mount your code in to an environment
  \item Edit locally in your env of choice
  \item Run the app/tests in your target env
  \end{itemize}
\end{frame}

\begin{frame}
  \frametitle{For CI}
  \begin{itemize}
  \item Run each test in it's own environment
  \item No more ``My selenium tests require FFox X.Y but the build slaves have X.Z''
  \end{itemize}
\end{frame}

\begin{frame}
  \frametitle{In your tests}
  \begin{itemize}
  \item Docker has simple rest api
  \item SetUp/TearDown containers as fixtures
  \end{itemize}
\end{frame}

\begin{frame}
  \frametitle{Docker vs Ansible/Chef/Salt/Puppet}
  \begin{itemize}
  \item Docker is focused on app packaging and distribution.
  \item Use Ansible to prepare the nodes, pull and start your containers.
  \end{itemize}
\end{frame}

\begin{frame}
  \frametitle{Or the big boys}
  \begin{itemize}
  \item Kubernetes
  \item CoreOS
  \item Docker Swarm
  \end{itemize}
\end{frame}

\begin{frame}
  \frametitle{Security}
  \begin{itemize}
  \item Isolation is not perfect!
  \item Don't just run everything as root!
  \item Privileged containers... the name says it all.
  \item Running untrusted images published on the docker hub is like downloading shareware in the 90s.
  \end{itemize}
\end{frame}

\end{document}
